\documentclass{article}

\usepackage{ctex}
\usepackage{graphicx}
\usepackage{float}
\usepackage{datetime}
\usepackage{amssymb}
\usepackage{setspace}
\usepackage{amsmath}
\usepackage{geometry}
\usepackage{tikz}
\usetikzlibrary{positioning} %为了实现相对位置的设定
\usepackage{xcolor} %为了实现不同的颜色

\geometry{left=3.0cm,right=2.0cm,top=2.5cm,bottom=2.5cm}
\title{离散数学作业 Problem Set 8}
\renewcommand{\baselinestretch}{1.5} %调整行间距

\author{201830099 周义植}
\date{\today}

\begin{document}
\maketitle
\section*{Problem 1}
a) 3,3,3,3

b) 2,2,2,2

c) 4,3,3,3

d) 3,3,3,3

\section*{Problem 2}
证明:设有n个顶点,因为简单图顶点度数最多为n-1,而各个顶点度数又不相同,所以所有顶点的度数刚好为0$\sim $n-1.但若有顶点的度数为0,则度最大的顶点也未连接该点,其度不可能为n-1,矛盾。

\section*{Probloem 3}
证明:$\delta (G)\leqslant d(v_i) \leqslant \varDelta (G)$

所以$\mathcal{V} \delta (G)\leqslant \Sigma d(v_i) \leqslant \mathcal{V} \varDelta (G)$

由握手定理,$ \delta (G)\leqslant \frac{2\epsilon}{\mathcal{V} } \leqslant n\varDelta (G)$
\section*{Problem 4}
\subsection*{1}
\begin{center}
    邻接矩阵A
\end{center}
\begin{equation}
    \begin{array}{c|c|c|c|c|c} 
        & v_{1} & v_{2} & v_{3} & v_{4} & v_{5} \\
        \hline v_{1} & 0 & 1 & 0 & 0 & 0 \\
        \hline v_{2} & 1 & 0 & 0 & 0 & 0 \\
        \hline v_{3} & 0 & 0 & 0 & 1 & 1 \\
        \hline v_{4} & 0 & 0 & 1 & 0 & 1 \\
        \hline v_{5} & 0 & 0 & 1 & 1 & 0
        \end{array}
\end{equation}

\begin{center}
    关联矩阵B
\end{center}
\begin{equation}
    \begin{array}{c|c|c|c|c} 
        & e_{1} & e_{2} & e_{3} & e_{4}\\
        \hline v_{1} & 1 & 0 & 0 & 0  \\
        \hline v_{2} & 1 & 0 & 0 & 0  \\
        \hline v_{3} & 0 & 1 & 0 & 1  \\
        \hline v_{4} & 0 & 1 & 1 & 0  \\
        \hline v_{5} & 0 & 0 & 1 & 1 
        \end{array}
\end{equation}

\begin{center}
    $BB^T-A$
\end{center}
\begin{equation}
    \begin{array}{c|c|c|c|c|c} 
        & v_{1} & v_{2} & v_{3} & v_{4} & v_{5} \\
        \hline v_{1} & 1 & 0 & 0 & 0 & 0 \\
        \hline v_{2} & 0 & 1 & 0 & 0 & 0 \\
        \hline v_{3} & 0 & 0 & 2 & 0 & 0 \\
        \hline v_{4} & 0 & 0 & 0 & 2 & 0 \\
        \hline v_{5} & 0 & 0 & 0 & 0 & 2
    \end{array}
\end{equation}

2)
D是对角矩阵,对角线上的元素表示每个节点的度。

$d_{ij}=d_{ji}=B_{i:}\cdot B_{j:}$(表示行向量),即$d_{ij}$表示$v_i,v_j$共同的边数,若$i\neq j$则表示两点间的边数(如简单图则为0or1,表示两点是否相连),特别地$d_{ii}$表示该点所连的边数,即为该点的度。$D-A$将$i\neq j$的元素减为零,只留下对角线上的元素表示每个点的度。
\section*{Problem 5}
证明:

如一个简单图G是自补图,设其边数为n,则$e(G)=e(\bar{G}),e(G)+e(\bar{G})=e(K_n)=\frac{n(n-1)}{2},\therefore e(G)=\frac{n(n-1)}{4},n\equiv 0,1(mod~ 4)$

又因为该图为正则图,设$d(v_i)=k$则对其补图也有$d(v_i)=k$,则每个点在完全图中的度为$d(v_i)=2k,\therefore 2 ~|~ n-1$

综上,$n\equiv 1(mod~ 4)$

\section*{Problem 6}
证明:

首先将一个图G的顶点度的平均值记作$z(G)$,去掉一个顶点后的图为$G^{'}$
\subsection*{a}
$z(G) = \frac{\sum\limits_{i=1}^{n-1}d(v_i)+\varDelta (G)}{n}$;删去度最大的顶点后,每个其余与其相邻顶点的度都减一。$z(G^{'})=\frac{\sum\limits_{i=1}^{n-1}d(v_i)-\varDelta (G)}{n-1}$

$n(n-1)(z(G)-z(G^{'}))=n\varDelta (G)-\sum\limits_{i=1}^{n-1}d(v_i)$

\subsection*{b}
反驳:对于$K_3,z(G)=2$删去某个点,$z(G^{'})=1.$
\section*{Problem 7}
证明:如图不连通,则其可分为几个连通子图。在G的补图中,每个连通子图中的每个点都会连到另一个连通子图的每个点,至少都会连到同一个点,所以这些点与该共同点互相连通。同理,所有的点都互相连通。

\section*{Problem 8}
\subsection*{1}
不是强连通的,因为点a出度为0。是弱连通的。

\subsection*{2}
不是强连通的,因为点c出度为0。是弱连通的。

\subsection*{3}
不是强连通的,也不是弱连通的。

\section*{Problem 9}
证明:

对于任意两点,若既互不相连也没有公共节点,则这两个点所连的点的集合不相交,则这两个集合的势之和最多为n-2(所有的点再减去这两个点),即两个点的度之和最多为n-2,矛盾。所以这两个点一定要么相连要么连接一个以上共同点,该两点连通。又由于点选取的任意性,该图连通。

\section*{Problem 10}
证明:

当n=1,2时显然成立。

假设当n=k时,结论成立,即对于有k个顶点的图,若$e>\frac{(k-1)(k-2)}{2}$时结论成立;

当n=k+1时,若此时每个结点度数为k,则结论显然成立,否则必存在一个结点v度数至多只有k-1,
即这个结点最多只有k-1条边和它相连。因为此时总的边数$e>\frac{k(k-1)}{2}$,则其它k个结点之间的边数$e^{'}>  \frac{k(k-1)}{2}-(k-1)=\frac{(k-1)(k-2)}{2}$。
根据归纳假设,这k个结点之间是连通的。

而对除去的这个节点,若其度为0,则除去该点后的图$e^{'}>\frac{k(k-1)}{2}=e(K_{k})$,显然矛盾。所以该点至少与其余点中的1个点连通,因此整个图连通。

由数学归纳法,结论成立。
\end{document}