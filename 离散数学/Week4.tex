\documentclass{article}

\usepackage{ctex}
\usepackage{graphicx}
\usepackage{float}
\usepackage{datetime}
\usepackage{amssymb}
\usepackage {amsmath}
\usepackage{geometry}

\geometry{left=3.0cm,right=2.0cm,top=2.5cm,bottom=2.5cm}
\title{离散数学作业}
\author{Yizhi Zhou}
\date{\today}

\begin{document}

\maketitle
\section*{Problem 1}

(1) card A = 3\\

(2)有双射$ f:N \rightarrow B : f(x) = x^{2}, card B = card N = \aleph_{0}$\\

(3)有双射$ f:N \rightarrow C : f(x) = x^{109}, card C = card N = \aleph_{0}$\\

(4)有双射$ f:N \rightarrow D : f(x) = x^{218}, card D = card N = \aleph_{0}$\\

(5)$B\cup C \approx N,cardE = \aleph_{0}$\\

(6)有双射$ f:R^{+} \rightarrow F : f(r) = (x^2+y^2=r^2) , card F = card R^{+} = \aleph$

\section*{Probelm 2}
(a) 可数无限的。有双射$f:N \rightarrow A$
$$ f(x)=x+11
$$

(b) 可数无限的。有双射$f:N \rightarrow B$
$$ f(x)=-2x-1
$$

(c) 有限的。

(d) 不可数的。

(e) 可数无限的。有双射$f:N \rightarrow E$
$$ f(x)=\left\{
    \begin{aligned}
    (2 , \frac{x}{2} + 1),x\equiv0(mod2) \\
    (3 , \frac{x-1}{2} + 1),x\equiv1(mod2)
    \end{aligned}
    \right.
$$

(f) 可数无限的。有双射$f:N \rightarrow F$
$$ f(x)=\left\{
    \begin{aligned}
    10 \cdot \frac{x}{2},x\equiv0(mod2) \\
    -10 \cdot \frac{x+1}{2},x\equiv1(mod2)
    \end{aligned}
    \right.
$$

\section*{Problem 3}
证明:

若有从A到B的满射,则对B中的每个元素,只取A中对应的某一个元素,再将关系反向,即可得到B到A的单射,因此A优势于B。
A可数,所以B一定可数。

\section*{Problem 4}
证明:

任何一个有理数$y$必能化为既约分数$\frac{p}{q}$的形式。因此我们可以找到满射$f:N^{2}\rightarrow Q:$
$$
f((p,q)) = \frac{p}{q}
$$
再从原点螺旋向外经过每一个整数点并排序即可得到$N\rightarrow Q$的满射,因此$N$优势于$Q$;

而$N \in Q$,所以$Q$优势于$N$,所以二者等势.
\section*{Problem 5}
证明:

$if\qquad   n\equiv0(mod k)$,则$\lceil \frac{n}{k} \rceil = \frac{n}{k},\lfloor \frac{n-1}{k} \rfloor = \frac{n}{k} - 1$

$elif\qquad   n\equiv 1,2,\cdots (k-1)(mod k)$,则$\lceil \frac{n}{k} \rceil = \frac{n}{k}+1,\lfloor \frac{n-1}{k} \rfloor = \frac{n}{k}$

两种情况下都有 $\lceil n / k\rceil=\lfloor(n-1) / k\rfloor+1$

\section*{Problem 6}

(a) 否。$a=6,b=15,c=30:a|c,b|c,ab\nmid c$

(b) 是。$a\mid c,$则存在整数$m\ s.t.\ \frac{c}{a} = m$同理存在整数$n\ s.t.\ \frac{d}{b}=n,\therefore\ \frac{cd}{ab}=mn,ab\mid cd$

(c) 是。$a\mid ab,ab\mid c,\therefore a\mid c,$

(d) 否。$a=4,b=2,c=2:a\mid bc\qquad but\qquad a\nmid b\quad and\quad a\nmid c $

\section*{Problem 7}
(a) 计算得 $23300\ mod\ 11 = 2$

(b) 首先有$2^{5}\equiv 1(mod31),\therefore 2^{3300} = (2^{5})^{660}\equiv 1^{660} = 1(mod31)$

(c) 由费马小定理 $3^{6}\equiv 1(mod7),\therefore 3^{516} = (3^{6})^{86}\equiv 1^{86} = 1(mod7)$

\section*{Problem 8}
证明:

(a)连续3个整数,其中必有一个使得$2\mid x$,一个$3\mid y$($xy$可能相同)

而$6=2\times 3,gcd(2,3)=1,\therefore 6\mid n(n+1)(n+2)$

(b)

\begin{equation}
    \begin{split}
        \frac{1}{5} n^{5}+\frac{1}{3} n^{3}+\frac{7}{15} n &= \frac{3n^{5}+5n^{3}+7n}{15}\\
        &= \frac{n(3n^{4}+5n^{2}+7)}{15}\\
        &= \frac{n(n+1)(n-1)(3n^{2}+8)+15}{15}
    \end{split}
\end{equation}

因此只需证明$15 \mid N = n(n+1)(n-1)(3n^{2}+8)$.

显然$3\mid n(n+1)(n-1)$.当$n\equiv 0,1,4(mod5)$时,均有$5\mid n(n+1)(n-1)$,从而得证;

当$n\equiv 2(mod5),3n^{2}+8\equiv 3*4+8\equiv 0(mod5)$,

当$n\equiv 3(mod5),3n^{2}+8\equiv 3*9+8\equiv 0(mod5)$,

而$15=3\times 5,gcd(3,5)=1,$

从而$15\mid n(n+1)(n-1)(3n^{2}+8)$.


\section*{Problem 9}

(a) 
\begin{equation} \begin{split}
    \notag
    125 &= 85 + 40 \\
      85 &= 40\times 2 + 5 \\
      40 &= 5\times 8
    \end{split} \end{equation}
$\therefore\ gcd(85,125)=5$

(b) 
\begin{equation} \begin{split}
    \notag
    231 &= 72\times  + 15 \\
        72 &= 15\times 4 + 12 \\
        15 &= 12 + 3\\
        12 &= 3\times 4
    \end{split} \end{equation}
$\therefore\ gcd(231,72)=3$

(c) 
\begin{equation} \begin{split}
    \notag
    56 &= 45  + 11 \\
        45 &= 11\times 4 + 1 \\
        11 &= 1\times 11\\
    \end{split} \end{equation}
$\therefore\ gcd(45,56)=1$

(d) 
\begin{equation} \begin{split}
    \notag
    154 &= 64\times2   + 26 \\
        64 &= 26\times 2 + 12 \\
        26 &= 12\times 2 + 2\\
        12 &= 2\times 6
    \end{split} \end{equation}
$\therefore\ gcd(154,64)=2$
\section*{Problem 10}
证明:

因为存在无限个素数,对充分大的素数n,

$\phi(n)=n-1,$而$(n+1)$至少有一个质因数2,$\phi(n+1)\leq (n+1)(1-\frac{1}{2})\textless\ \phi(n)$

因此存在无限个$n,s.t.\ \phi(n)>\phi(n+1)$
\end{document}