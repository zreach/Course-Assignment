\documentclass{article}

\usepackage{ctex}
\usepackage{graphicx}
\usepackage{float}
\usepackage{datetime}
\usepackage{amssymb}
\usepackage{setspace}
\usepackage{amsmath}
\usepackage{geometry}
\usepackage{tikz}
\usetikzlibrary{positioning} %为了实现相对位置的设定
\usepackage{xcolor} %为了实现不同的颜色

\geometry{left=3.0cm,right=2.0cm,top=2.5cm,bottom=2.5cm}
\title{离散数学作业 Problem Set 7}
\renewcommand{\baselinestretch}{1.5} %调整行间距

\author{201830099 周义植}
\date{\today}

\begin{document}
\maketitle
\section*{Problem 1}
(1)
有限循环群的生成元为$a^r$,其中$gcd(a,r)=1$,所以所有生成元为$a,a^{1},a^{2},a^{4},a^{7},a^{8},a^{11},a^{13},a^{14}$

(2)
1阶子群 $\{e\}$

3阶子群 $\{e,a^{5},a^{10}\}$

5阶子群 $\{e,a^{3},a^{6},a^{9},a^{12}\}$

15阶子群G

\section*{Problem 2}
证明:

设循环群的生成元为a,对任意两个元素$a^m,a^n,a^m\times a^n = a^{m+n} = a^n\times a^m$,满足交换律,为阿贝尔群。

不一定。可找出反例,如Klevin 四元群。

\section*{Problem 3}
证明:

设a为$G_1$生成元,

对$\forall p \in f(G_1),$有$x\in G_1, x=a^r$

$\therefore p=f(a^r)=f^r(a),$所以$f(G)$为循环群,有生成元$f(a)$。

\section*{Problem 4}
(a)是。

(b)不是。对于\{c,b\},其上界有\{d,e,f\},而无最小上界。

(c)是。

(d)不是。对于\{b,c\},其上界有\{d,e,f,g\},无最小上界。

(e)不是。对于\{d,e\},无下界。

(f)是。

\section*{Problem 5}
(a)\{a,d\}互补。c,d没有补元。

(c)\{a,f\},互补。b:c,d。c:b,e。d:b,e。e:c,d。

(f)\{a,f\},互补。b:e.e:b.其余元素没有补元。

\section*{Problem 6}
\subsection*{a}
是分配格。因为小于五元。

不是有补格,因为b,c没有补元。

所以其不是布尔格。

\subsection*{c}
不是分配格。$b\wedge c=b\wedge d,b\vee c=b\vee d,but\ c\neq d.$由分配格的判定定理2知其不为分配格。

是有补格,由上一题知每个元素都有其补元。

所以其不是布尔格。

\subsection*{f}
是分配格。因为没有与钻石格或五角格同构的子格。

由上一问知不是有补格。

所以不是布尔格。
\section*{Problem 7}
证明:

设a为生成元,$x=a^{r},r=0,1,\cdots n-1,$则等价于求解r。

$x^m=e \Leftrightarrow (a^r)^{m}=a^{kn},k\in N$,也即$r=\frac{kn}{m}$

当$k=0,1,\cdots m-1$时,r的解都满足条件且互不相等,所以方程共有m个解。

\section*{Problem 8}
证明:

由交换律,$(ab)^{mn}=a^{mn}b^{mn}=e.$设$|ab|=k$,显然$k|mn.$

又因为$gcd(m,n)=1,$所以k=1或m或n或mn。

若k=1,则ab=e,$(ab)^{m}=b^m=e,$与$b^n=e,(m,n)=1$矛盾(除非n=1),同理也可推出m=1,此时mn=1

若k=m,则$(ab)^{m}=b^m=e,$与$b^n=e,(m,n)=1$矛盾(除非n=1,此时mn=m。)

k=n同理。

上述中的特殊情况也等价于mn,所以k只能等于mn。
\section*{Problem 9}
证明:

首先,显然$S\subseteq L.$

$\forall x,y\in S,$设$d\in L\wedge d\notin S,x\vee y=d,$则a,d均为\{x,y\}上界,而$d\npreceq a,$则d肯定不为x,y的上确界,矛盾。所以$x\vee y\in S.$

再令$f=x\wedge y,$则$f\preceq x,$又已知$x\preceq a,\therefore f\preceq a,f\in S.$

综上所述,S满足L上的交并运算,因此$<S,\preceq>$为L的子格。

\section*{Problem 10}
能。显然该运算封闭。

同时,满足结合律:$\forall x,y,z\in B,$

\begin{equation}
    \nonumber
    \begin{aligned}
        &(x \oplus y) \oplus z=\left(\left(x \wedge y^{\prime}\right) \vee\left(x^{\prime} \wedge y\right)\right) \oplus z \\
        =&\left(\left(\left(x \wedge y^{\prime}\right) \vee\left(x^{\prime} \wedge y\right)\right) \wedge z^{\prime}\right) \vee\left(\left(\left(x \wedge y^{\prime}\right) \vee\left(x^{\prime} \wedge y\right)\right)^{\prime} \wedge z\right) \\
        =&\left(x \wedge y^{\prime} \wedge z^{\prime}\right) \vee\left(x^{\prime} \wedge y \wedge z^{\prime}\right) \vee\left(\left(x \wedge y^{\prime}\right)^{\prime} \wedge\left(x^{\prime} \wedge y\right)^{\prime} \wedge z\right) \\
        =&\left(x \wedge y^{\prime} \wedge z^{\prime}\right) \vee\left(x^{\prime} \wedge y \wedge z^{\prime}\right) \vee\left(\left(x^{\prime} \vee y\right) \wedge\left(x \vee y^{\prime}\right) \wedge z\right) \\
        =&\left(x \wedge y^{\prime} \wedge z^{\prime}\right) \vee\left(x^{\prime} \wedge y \wedge z^{\prime}\right) \vee\left(\left(\left(x^{\prime} \wedge x\right) \vee\left(x^{\prime} \wedge y^{\prime}\right) \vee(x \wedge y) \vee\left(y \wedge y^{\prime}\right)\right) \wedge z\right) \\
        =&\left(x \wedge y^{\prime} \wedge z^{\prime}\right) \vee\left(x^{\prime} \wedge y \wedge z^{\prime}\right) \vee\left(x^{\prime} \wedge y^{\prime} \wedge z\right) \vee(x \wedge y \wedge z)
        \end{aligned}
\end{equation}

同理也可化简

$(x \oplus y) \oplus z=\left(\left(x \wedge y^{\prime}\right) \vee\left(x^{\prime} \wedge y\right)\right) \oplus z =  \left(x \wedge y^{\prime} \wedge z^{\prime}\right) \vee\left(x^{\prime} \wedge y \wedge z^{\prime}\right) \vee\left(x^{\prime} \wedge y^{\prime} \wedge z\right) \vee(x \wedge y \wedge z) $

又$0 \oplus y=\left(0 \wedge y^{\prime}\right) \vee\left(1 \wedge y\right)=0\vee y=y = y\oplus 0$所以0为单位元。

$\forall x\in B,x\oplus x=(x\vee x)\wedge (x^{'}\vee x^{'})=x\wedge x^{'}=0$,所以每个元素都有其逆元即本身。

所以能构成群。又显然满足交换律,所以为阿贝尔群。
\section*{Problem 11}
证明:

(1)

对n进行归纳。n=2时即为德摩根率成立。当n=k成立时,

\begin{equation}
    \nonumber
    \begin{aligned}
        &\left(a_{1} \vee a_{2} \vee \cdots \vee a_{k+1}\right)^{\prime}=\left(\left(a_{1} \vee a_{2} \vee \cdots \vee a_{k}\right) \vee a_{k+1}\right)^{\prime}=\left(a_{1} \vee a_{2} \vee \cdots \vee a_{k}\right)^{\prime} \wedge a_{k+1}{ }^{\prime} \\
        =&\left(a_{1}^{\prime} \wedge a_{2}^{\prime} \wedge \cdots \wedge a_{k}^{\prime}\right) \wedge a_{k+1}{ }^{\prime}=a_{1}^{\prime} \wedge a_{2}^{\prime} \wedge \cdots \wedge a_{k}^{\prime} \wedge a_{k+1}^{\prime}
        \end{aligned}
\end{equation}

n=k+1成立。

(2)

即为(1)的对偶命题,显然成立。

\section*{Problem 12}
证明:

\begin{equation}
    \nonumber
    \begin{aligned}
        v \vee\left(u \wedge w^{\prime}\right) &=(w \wedge x) \vee\left((w \vee x) \wedge w^{\prime}\right) \\
        &=(w \wedge x) \vee\left(\left(w \wedge w^{\prime}\right) \vee\left(x \wedge w^{\prime}\right)\right) \\
        &=(w \wedge x) \vee\left(0 \vee\left(x \wedge w^{\prime}\right)\right) \\
        &=(w \wedge x) \vee\left(w^{\prime} \wedge x\right) \\
        &=x \wedge\left(w \vee w^{\prime}\right) \\
        &=x \wedge 1 \\
        &=x
        \end{aligned}
\end{equation}
\end{document}