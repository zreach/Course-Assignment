\documentclass{article}

\usepackage{ctex}
\usepackage{graphicx}
\usepackage{float}
\usepackage{datetime}
\usepackage{amssymb}
\usepackage{setspace}
\usepackage{amsmath}
\usepackage{geometry}
\usepackage{tikz}
\usetikzlibrary{positioning} %为了实现相对位置的设定
\usepackage{xcolor} %为了实现不同的颜色
\usepackage{listings}
\usepackage{qtree}% 绘制语法分析树
\usepackage{forest}% 绘制语法分析树
\lstset{
    numbers=left, 
    numberstyle= \tiny, 
    keywordstyle= \color{ blue!70},
    commentstyle= \color{red!50!green!50!blue!50}, 
    frame=shadowbox, % 阴影效果
    rulesepcolor= \color{ red!20!green!20!blue!20} ,
    escapeinside=``, % 英文分号中可写入中文
    xleftmargin=2em,xrightmargin=2em, aboveskip=1em,
    framexleftmargin=2em
} 

\geometry{left=3.0cm,right=2.0cm,top=2.5cm,bottom=2.5cm}
\title{离散数学 Problem Set 11}
\renewcommand{\baselinestretch}{1.5} %调整行间距
\tikzstyle{startstop} = [rectangle, rounded corners, minimum width=3cm, minimum height=1cm,text centered, draw=black]
\tikzstyle{decision} = [diamond, draw, text width=5.5em, text badly centered, inner sep=0pt]
\tikzstyle{process} = [rectangle, minimum width=3cm, minimum height=1cm, text centered, draw=black]
\tikzstyle{arrow} = [thick,->,>=stealth]
\tikzset{
    bnode/.style ={
    circle, %矩形节点
    minimum width =20pt, %最小宽度
    minimum height =20pt, %最小高度
    inner sep=5pt, %文字和边框的距离
    draw=blue, %边框颜色}
    fill=black,
    text=white
    }
}
\tikzset{
    rnode/.style ={
    circle, %矩形节点
    minimum size =20pt, %最小宽度
    % max size = 25pt,
    inner sep=5pt, %文字和边框的距离
    draw=blue, %边框颜色}
    fill=red,
    text=white
    }
}
\tikzset{
    leaf/.style={
    rectangle,
    minimum size =20pt, %最小宽度
    inner sep=5pt, %文字和边框的距离
    draw=blue, %边框颜色}
    fill=white,
    text=black
    }
}
\tikzset{
    fa/.style={
    circle,
    minimum size =20pt, %最小宽度
    inner sep=5pt, %文字和边框的距离
    draw=blue, %边框颜色}
    fill=white,
    text=black
    }
}
\tikzset{
    enode/.style ={
    circle, %矩形节点
    minimum size =5pt, %最小宽度
    inner sep=5pt, %文字和边框的距离
    draw=blue, %边框颜色}
    fill=black,
    text=white
    }

    
}

\lstset{numbers=left,showstringspaces=false,frame=single}
\geometry{left=2.50cm, right=2.50cm, top=2cm, bottom=2cm}
\author{201830099 周义植}
\date{\today}

\begin{document}
\maketitle
\section*{Problem 1}
\subsection*{a}
\begin{center}
\begin{tikzpicture}[ 
    level 1/.style={sibling distance=10cm, level distance=1.5cm},
    level 2/.style={sibling distance=5cm, level distance=1.5cm},
    level 3/.style={sibling distance=3cm, level distance=1.5cm},
    ]
    \node[fa]{}
    child {
        node[leaf] {a(0.36)}
    }
    child{
        node[fa]{}
        child{
            node[fa]{}
            child{
                node[leaf]{b(0.18)}
            }
            child{
                node[leaf]{c(0.18)}
            }
        }
        child{node[fa]{}
        child{
            node[leaf]{d(0.1)}
        }
        child{
            node[fa]{}
            child{
                node[leaf]{e(0.08)}
            }
            child{
                node[fa]{}
                child{
                    node[leaf]{f(0.06)}
                }
                child{
                    node[leaf]{g(0.04)}
                }
            }
        }
        }
    };
\end{tikzpicture}
\end{center}

\begin{center}
    \begin{tikzpicture}[ 
        level 1/.style={sibling distance=10cm, level distance=1.5cm},
        level 2/.style={sibling distance=5cm, level distance=1.5cm},
        level 3/.style={sibling distance=2cm, level distance=1.5cm},
        ]

    \node[fa]{}
    child{
        node[fa]{}
        child{
            node[leaf]{b(0.18)}
        }
        child{
            node[fa]{}         
            child{
                node[leaf]{d(0.10)}
            }
            child{
                node[leaf]{e(0.08)}
            }
        }
    }
    child{
        node[fa]{}
        child{
            node[leaf]{a(0.36)}
        }
        child{
                node[fa]{}
                child{
                    node[leaf]{c(0.18)}
                }
                    child{
                        node[fa]{}
                        child{
                            node[leaf]{f(0.06)}
                        }
                        child{
                            node[leaf]{g(0.04)}
                        }
                    }
        }
    };
    \end{tikzpicture} 
\end{center}
\subsection*{b}
1) 平均:2.56 方差:1.7264

2) 平均:2.56 方差:0.4464

第二种方法。

\section*{Prolem 2}

\subsection*{a}
8

\subsection*{b}
21

\subsection*{c}
32

\subsection*{d}
32

\subsection*{e}
2048

\subsection*{f}
27

\section*{Problem 3}
\subsection*{Prim(从点A开始)}

定义:状态中的tree,unseen,fringe分别表示该点已被加入T、还未发现以及加入备选方案。

加入点A:

\begin{equation}
    \nonumber
    \begin{array}{l}
        \begin{array}{lllllllllll}
        \hline & A & B & C & D & E & F & G & H & I \\
        \hline \text { fringewgt } & 0 & 5 &  & 2 & &  &  &  &  \\
        \text { parent } & -1  & A &  & A & &  &  &  & \\
        \text { status } & tree & fringe & unseen & fringe & unseen & unseen  &  unseen  & unseen & unseen
        \end{array}\\
        \hline
    \end{array}
\end{equation}

选择AD:

\begin{equation}
    \nonumber
    \begin{array}{l}
        \begin{array}{lllllllllll}
        \hline & A & B & C & D & E & F & G & H & I \\
        \hline \text { fringewgt } & 0 & 3 &  & 2 & 7 &  & 6 &  &  \\
        \text { parent } & -1  & D &  & A & D &  & D &  & \\
        \text { status } & tree & fringe & unseen & tree & fringe & unseen  &  fringe  & unseen & unseen
        \end{array}\\
        \hline
    \end{array}
\end{equation}

以此类推,选择DB,BC,CF,FE,EH,HI,HG.

\subsection*{Kruskal}

加边:EF 并查集: (EF)
加边:AD 并查集: (EF) (AD)
加边:HI 并查集: (EF) (AD) (HI)
加边:BD 并查集: (EF) (ABD) (HI)
加边:CF 并查集: (CEF) (ABD) (HI)
加边:HE 并查集: (CEFHI) (ABD)
加边:BC 并查集: (ABDCEFHI)
加边:GH 并查集: (ABDCEFHIG)



\section*{Problem 4}
3次。

第一次,1,2,3与4,5,6。有两种可能的结果:

1)一样重。则伪币在7,8,将1与7称重,如不平则7为伪币,反之为8;

2)不一样重。由对称性,不妨1,2,3轻。将1,7与4,5称重,如不平衡,只能是1,7轻。此时有3种可能:1轻或4/5重。则再将4,5称重,如不平衡则重的为伪币,如平衡则1为伪币。
如1,7与4,5等重,则可能性为2/3轻或6重,则将2,3称重,轻的为伪币,等重则6为伪币。

\section*{Problem 5}
\begin{figure}[H]
    \centering
    \includegraphics*[scale=0.2]{p1.jpg}
\end{figure}
\section*{Problem 6}
证明:

对于该边较晚发现的点A来说,若要该边不是背边,则需要另一个点B在此时已完全结束(否则A一定是B的后代)。而在遍历B的整个过程中,A都没有开始遍历,那么一定会遍历这条边,矛盾;
因此该边一定是背边。

\section*{Problem 7}
证明:

假设该图存在两个不同的最小生成树$T_1,T_2$,其边按权值的升序排列分别为$e_1,e_2,\cdots e_k$与$e^{\prime}_1,e^{\prime}_2,\cdots e^{\prime}_k$。由于边的权重互不相同,每个权值唯一地代表一条边。
设$i$是最小的使$e_i\neq e^{\prime}_i$的下标,不失一般性令$e_i< e^{\prime}_i$,则对$T_2+\{e_i\}$必定存在圈,且其中存在不在$T_1$中的边ab,则$W(ab)\geq W(e^{\prime}_i) > W(e_i)$,因此用$e_i$代替ab可得到权值更小的生成树,与假设的MST矛盾。
因此边权值互不相同的图MST唯一。
\end{document}