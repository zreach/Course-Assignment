\documentclass{article}

\usepackage{ctex}
\usepackage{graphicx}
\usepackage{float}
\usepackage{datetime}
\usepackage{amssymb}
\usepackage{setspace}
\usepackage{amsmath}
\usepackage{geometry}
\usepackage{tikz}
\usetikzlibrary{positioning} %为了实现相对位置的设定
\usepackage{xcolor} %为了实现不同的颜色

\geometry{left=3.0cm,right=2.0cm,top=2.5cm,bottom=2.5cm}
\title{离散数学作业 Problem Set 6}
\renewcommand{\baselinestretch}{1.5} %调整行间距

\author{201830099 周义植}
\date{\today}

\begin{document}
\maketitle
\section*{Problem 1}
\subsection*{$f_1$}
可交换:$x+y=y+x$

可结合:$x+(y+z)=(x+y)+z$

不幂等:只有$0+0=0,x\neq 0$时,$x+x\neq x$

单位元:0,$0+x=x+0=x$

零元:无

逆元:$x+(-x)=0,\therefore x^{-1}=-x$

\subsection*{$f_2$}
不可交换:$x\neq y$时$x-y\neq y+x$

可结合:$x-(y-z)=(x-y)-z$

不幂等:$x\neq 0$时,$x-x\neq x$

单位元不存在,但是有右单位元0:$x-0=x$

零元:无

逆元:不存在单位元,所以不讨论。

\subsection*{$f_3$}
可交换:$x\cdot y=y\cdot x$

可结合:$x\cdot (y\cdot z)=(x\cdot y)\cdot z$

不幂等:$x\neq 0,1$时,$x\cdot x\neq x$

单位元:1,$1\cdot x=x\cdot 1 = x$

零元:0,$0\cdot x=x\cdot 0=0$

逆元:0没有逆元,除此之外$x\cdot \frac{1}{x}=1,\therefore x^{-1}=\frac{1}{x}(x\neq 0)$

\subsection*{$f_4$}
可交换:$max(x,y)=max(y,x)$

可结合:$max(x,max(y,z))=max(max(x,y),z)=max(x,y,z)$

幂等:$max(x,x)=x$

单位元:不存在。

零元:不存在。

逆元:不讨论。

\subsection*{$f_5$}
可交换:$min(x,y)=min(y,x)$

可结合:$min(x,min(y,z))=min(min(x,y),z)=min(x,y,z)$

幂等:$min(x,x)=x$

单位元:不存在。

零元:不存在。

逆元:不讨论。

\subsection*{$f_6$}
可交换:$|x-y|=|y-x|$

不可结合:$x=2,y=1,z=3$时,$|x-|y-z||=0,||x-y|-z|=2$

不幂等:$x\neq 0$时,$|x-x|\neq x$

单位元:不存在。$x<0$时,$\forall a\in R,|a-x|=|x-a|\neq x$

零元:不存在。

逆元:不讨论。
\section*{Problem 2}
1.
能。

交换律:$\forall x,y\in S,gcd(x,y)=gcd(y,x)$

结合律:$\forall x,y,z\in S,gcd(x,gcd(y,z))=gcd(gcd(x,y),z)=gcd(x,y,z)$

单位元:不存在。

零元:1.$\forall x\in S,gcd(1,x)=gcd(x,1)=1$

2.不能,因为该运算在S上不封闭。反例:$lcm(7,9)=63$

3.能。

交换律:$\forall x,y\in S,x*y=y*x = max(x,y)$

结合律:$\forall x,y,z\in S,(x*y)*z=x*(y*z)=max(x,y,z)$

单位元;1.$forall x\in S,1*x=x*1=x$

零元:10.$forall x\in S,10*x=x*10=10$

4.不能。该运算在S上不封闭。$2*2=0$
\section*{Problem 3}
\subsection*{1}
是代数系统。满足交换律和结合律(和加法的性质相同)即$(g+f)(x)=g(x)+f(x)=f(x)+g(x)=(f+g)(x),((f+g)+p)(x)=(f+g)(x)+p(x)=f(x)+g(x)+p(x)=f(x)+(g+p)(x)=(f+(g+p))$。单位元为$f:f(x)=0,\forall x\in [a,b]$,没有零元。

\subsection*{2}
是代数系统。不满足交换律和结合律(和减法的性质相同)。没有单位元和零元。

\subsection*{3}
是代数系统。满足交换律和结合律(和乘法的性质相同,分析同第一小问)。单位元为$f_1:f_1(x)=1,\forall x\in [a,b]$零元为$f_0(x):f_0(x)=0,\forall x\in [a,b]$

\subsection*{4}
不构成代数系统。

\section*{Problem 4}
证明:

a)$a*b=a*(a*a)=(a*a)*a=b*a$

b)若$b*b=a$


\textcircled{1}若$a*b=b*a=a,then$
\begin{equation}
    \nonumber
    \begin{split}
        b*b&=a\\
        b*b*a&=a*a\\
        b*(b*a)&=a*a\\
        b*a&=b
    \end{split}
\end{equation}

与$b*a=a$矛盾;

\textcircled{2}若$a*b=b*a=b,then$
\begin{equation}
    \nonumber
    \begin{split}
        b*b&=a\\
        b*a*a&=a\\
        b*a&=a\\    
    \end{split}
\end{equation}

与$b*a=b$矛盾;

因此$b*b=b$
\section*{Problem 5}
证明:

若$\exists x\in G,x*x=x,$则$\ x=e*x=x^{-1}*x*x= x^{-1}*x = e$

\section*{Problem 6}
证明:

封闭性:$\forall x,y\in Z,$显然$x+y-2\in Z$;

结合律:$\forall x,y,t\in Z,(x+y-2)+t-2=x+(y+t-2)-2$;

单位元:$\forall x\in Z,x+2-2=2+x-2x,\therefore 1_s=2$;

逆元:$\forall x\in Z,$令$t=4-x,$知$t\in Z,$且$x+t-2=t+x-2=2.$所以对任意x,都存在其逆元t。

综上,G关于$\odot$构成群。

\section*{Problem 7}
证明:

首先由$ea=ae,e\in H$知H非空。之后

\textcircled{1}
$\forall x,y\in H,ax=xa,ay=ya,\therefore (xy)a=x(ya)=x(ay)=(xa)y=(ax)y=a(xy),xy\in H$

\textcircled{2}
$\forall x\in H,xa=ax$


\begin{equation}
    \nonumber
    \begin{split}
        x^{-1}xa&=x^{-1}ax\\
        a&=x^{-1}ax\\
        ax^{-1}&=x^{-1}axx^{-1}\\
        ax^{-1}&=x^{-1}a\\
    \end{split}
\end{equation}

$\therefore x^{-1}\in H$

由子群判定定理,知H为G子群。

\section*{Problem 8}
1,2,4,7,8,11,13,14与15互素,所以G的生成元有$a,a^{2},a^{4},a^{7},a^{8},a^{11},a^{13},a^{14}$

所有子群:

$<a>=G;$

$<a^{3}>=\{a^{3},a^{6},a^{9},a^{12},a^{15}\}$

$<a^{5}>=\{a^{5},a^{10},a^{15}\}$

\section*{Problem 9}
证明:

首先因H非空,所以$xHx^{-1}$一定非空。又因$x,h\in G,\therefore xhx^{-1}\in G,$H为G非空子集。

\textcircled{1}
$\forall h_{1},h_{2}\in H,(xh_{1}x^{-1})(xh_{2}x^{-1})=xh_{1}h_{2}x^{-1},$因为H为G的子群,所以$h_{1}h_{2}\in H,\therefore xh_{1}h_{2}x^{-1}\in xHx^{-1}$

\textcircled{2}
$\forall h \in H,h^{-1}\in H,\therefore xh^{-1}x^{-1}\in xHx^{-1}$

综上,$xHx^{-1}$为G的子群。

\section*{Problem 10}
证明:

(不妨把函数写作f与g)即证明$\forall x,y\in A,g(f(x\circ y))=g(f(x))\cdot g(f(y))$

首先有$\forall x,y\in A,f(x\circ y)=f(x)*f(y);\forall p,q\in B,g(p*q)=g(p)\cdot g(q)$

令$p=f(x),q=f(y),$代入即可得证。

\end{document}